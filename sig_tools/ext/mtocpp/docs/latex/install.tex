Make sure you have the latest version of mtoc++, see \hyperlink{download}{Downloading mtoc++}. Next step after installation is \hyperlink{tools}{Configuration and use of mtoc++}\hypertarget{install_inst_req}{}\subsection{Software requirements and recommendations}\label{install_inst_req}
The following programs need to be available on your machine in order to use mtoc++\+:
\begin{DoxyItemize}
\item {\ttfamily doxygen} ($>$=1.\+8.\+1)\+: mtoc++ is a filter for doxygen. If not yet available, get it at \href{http://www.doxygen.org}{\tt http\+://www.\+doxygen.\+org}
\end{DoxyItemize}

The following programs will highly improve your documentation creation experience if available\+:
\begin{DoxyItemize}
\item {\ttfamily dot\+:} A Graphviz tool that allows doxygen to create nice graphics for inheritance trees and collaboration diagrams.
\item {\ttfamily latex\+:} Required to use La\+Te\+X processing capabilities of doxygen (e.\+g. \href{http://www.latex-project.org/ftp.html}{\tt http\+://www.\+latex-\/project.\+org/ftp.\+html}). mtoc++ comes with some markups for better latex inclusion into the text flow. Also, easy inclusion of external latex sources and styles is included in mtoc++\textquotesingle{}s tools.
\item {\ttfamily ghostscript\+:} If using formulas with doxygen and you are not using pdflatex or are on a windows machine, this is a prerequisite (see \href{http://www.stack.nl/~dimitri/doxygen/install.html#install_bin_windows}{\tt http\+://www.\+stack.\+nl/$\sim$dimitri/doxygen/install.\+html\#install\+\_\+bin\+\_\+windows})
\end{DoxyItemize}

If you want to build mtoc++ from source, you will also need\+:
\begin{DoxyItemize}
\item Ragel\+: A finite-\/state machine compiler. Get at \href{http://www.complang.org/ragel}{\tt http\+://www.\+complang.\+org/ragel}
\item C\+Make\+: Cross-\/platform make. Get at \href{http://www.cmake.org}{\tt http\+://www.\+cmake.\+org}
\item dirent.\+h (We included a Visual Studio A\+P\+I implementation by Tony Ronkko for Windows)
\end{DoxyItemize}\hypertarget{install_inst_binaries}{}\subsection{Using precompiled binaries}\label{install_inst_binaries}
\hypertarget{install_inst_binaries_win}{}\subsubsection{Windows users}\label{install_inst_binaries_win}
If you are a windows user you can directly download the binaries at \hyperlink{download}{Downloading mtoc++}. Then simply place the binaries in a folder of your choice and add the folder to the P\+A\+T\+H environment variable. If you intend to use the \hyperlink{class_matlab_doc_maker}{Matlab\+Doc\+Maker}, you can also copy the mtoc++ binaries into the \char`\"{}documentation configuration files\char`\"{} folder for your/each project, this path will be added to P\+A\+T\+H by Mat\+Lab locally.

\begin{DoxyNote}{Note}
Depending on your system setup, you might need to install the Microsoft Visual C++ 2010 redistributables, which can be found \href{http://www.microsoft.com/download/en/confirmation.aspx?id=8328}{\tt here}.
\end{DoxyNote}
\begin{DoxyAttention}{Attention}
mtoc++ as well as doxygen expect all required programs (see \hyperlink{install_inst_req}{Software requirements and recommendations}) to be available via the P\+A\+T\+H environment variable, e.\+g. {\ttfamily latex.\+exe} or {\ttfamily gswin32c.\+exe} must be present in order for doxygen to work with La\+Te\+X output. Make sure that you have all requirements available, otherwise doxygen or the \hyperlink{class_matlab_doc_maker}{Matlab\+Doc\+Maker} will complain soon enough. You can check/change your Windows P\+A\+T\+H environment variable via the sequence 
\begin{DoxyCode}
Computer \(\backslash\) Properties \(\backslash\) Advanced system settings \(\backslash\) Environment Variables \(\backslash\) Edit Path 
\end{DoxyCode}

\end{DoxyAttention}
We are trying to always compile current Windows binaries for {\ttfamily mtocpp} and {\ttfamily mtocpp\+\_\+post} and include them for direct download.\hypertarget{install_inst_binaries_unix}{}\subsubsection{Unix binaries}\label{install_inst_binaries_unix}
For unix users we recommend to compile the sources following \hyperlink{install_inst_comp}{Compiling mtoc++ from source}.

However, we also plan to provide some precompiled linux binaries/packages soon. If you find a matching choice you can use it and all you have to do is to ensure that the binaries can be found on the environment P\+A\+T\+H.\hypertarget{install_inst_comp}{}\subsection{Compiling mtoc++ from source}\label{install_inst_comp}
Please check the \hyperlink{install_inst_req}{Software requirements and recommendations} when you intend to build mtoc++ yourself.

mtoc++ is built using the cmake (cross-\/platform make) tool. This tool is available for both unix and Windows, however, we only tested compiling our sources on linux and M\+S Visual Studio 2010.\hypertarget{install_inst_comp_win}{}\subsubsection{Windows platforms}\label{install_inst_comp_win}
For Windows compilation, you need a Windows C++ compiler (e.\+g. Min\+G\+W or Visual Studio). Then running the C\+Make G\+U\+I allows you to choose a compiler, specify any C\+Make configuration settings and create the makefiles/\+Visual Studio projects needed for compilation.

Furthermore, we\textquotesingle{}re using the {\ttfamily dirent.\+h} library for file access. As this is a linux library we\textquotesingle{}ve included a file {\ttfamily dirent\+\_\+msvc.\+h} in our source, which implements the dirent api for Microsoft Visual Studio and was written by Tony Ronkko. More information and downloads can be found at \href{http://www.softagalleria.net/dirent.php}{\tt http\+://www.\+softagalleria.\+net/dirent.\+php}.

\begin{DoxyNote}{Note}
On Windows, you can build both 32bit and 64bit versions. If you build with Visual Studio, in recent C\+Make versions you need to specify the target architecture already when choosing the generator (\char`\"{}\+Visual Studio 10 / Visual Studio 10 Win64\char`\"{}). This sets up the V\+S2010 project with the correct platforms. In general, you can of course also use 64bit binaries from ragel and doxygen, but this is not required for successful 64bit-\/compilation of mtoc++ .
\end{DoxyNote}
\hypertarget{install_inst_comp_unix}{}\subsubsection{Unix platforms}\label{install_inst_comp_unix}
The following procedure is an example of how to compile mtoc++ on a linux machine\+: 
\begin{DoxyCode}
 tar -xcvf mtocpp.tar.gz
cd mtocpp

\textcolor{preprocessor}{# Create build folder (optional, but more clean)}
mkdir build
cd build

\textcolor{preprocessor}{# Run cmake}
cmake ..
make install
\end{DoxyCode}


\begin{DoxyAttention}{Attention}
Please be aware that, depending on your installation location, you might need different access/write permissions. For most cases, a
\begin{DoxyCode}
sudo make install
\end{DoxyCode}
 will do the job if the above snippet fails.
\end{DoxyAttention}
\hypertarget{install_inst_comp_apple}{}\subsubsection{Apple hints}\label{install_inst_comp_apple}
For installation under recent Apple O\+S like 10.\+8.\+2, the \href{http://www.macports.org/}{\tt Mac\+Ports} project is a very useful tool to obtain prerequisites for mtoc++ compilation. Once installed, get {\ttfamily ragel} and {\ttfamily doxygen} via 
\begin{DoxyCode}
sudo port
> install ragel
> install doxygen
\end{DoxyCode}
\hypertarget{install_inst_cust}{}\subsubsection{C\+Make options\+: Installation folders and customization}\label{install_inst_cust}
\begin{DoxyNote}{Note}
These options are explained for the linux case, for windows the C\+Make G\+U\+I allows to set the relevant options.
\end{DoxyNote}
The default value for the install prefix is {\ttfamily /usr/local}, so the mtocpp binaries {\ttfamily mtocpp} and {\ttfamily mtocpp\+\_\+post} go to {\ttfamily /usr/local/bin} and the documentation is created inside {\ttfamily /usr/local/share/doc/mtocpp}.

If you want the \char`\"{}make install\char`\"{} command to copy the binaries and documentation to different locations, you can choose them by setting the following variables\+:
\begin{DoxyItemize}
\item C\+M\+A\+K\+E\+\_\+\+I\+N\+S\+T\+A\+L\+L\+\_\+\+P\+R\+E\+F\+I\+X\+: Set this to whatever location you want mtoc++ to be installed. Note that the binaries are effectively copied into \char`\"{}\+C\+M\+A\+K\+E\+\_\+\+I\+N\+S\+T\+A\+L\+L\+\_\+\+P\+R\+E\+F\+I\+X/bin\char`\"{} in order to comply with linux standards.
\item C\+U\+S\+T\+O\+M\+\_\+\+D\+O\+C\+\_\+\+D\+I\+R\+: This value is \char`\"{}\+C\+M\+A\+K\+E\+\_\+\+I\+N\+S\+T\+A\+L\+L\+\_\+\+P\+R\+E\+F\+I\+X/share/doc/mtocpp\char`\"{} per default.
\end{DoxyItemize}

So typing 
\begin{DoxyCode}
cmake -DCMAKE\_INSTALL\_PREFIX=\textcolor{stringliteral}{"/my/root/dir"} -DCUSTOM\_DOC\_DIR=\textcolor{stringliteral}{"/my/docs"} 
\end{DoxyCode}
 will copy the binaries to {\ttfamily /my/root/dir/bin} and the documentation to {\ttfamily /my/docs}.

If you left the {\ttfamily C\+U\+S\+T\+O\+M\+\_\+\+D\+O\+C\+\_\+\+D\+I\+R} flag empty the documentation would have gone to {\ttfamily /my/root/dir/share/doc/mtocpp} \hypertarget{install_inst_test}{}\subsection{Testing}\label{install_inst_test}
mtoc++ comes with some unit tests to check for e.\+g. successful compilation. Run the tests by typing 
\begin{DoxyCode}
make test 
\end{DoxyCode}
 in the same folder where you called {\ttfamily cmake}.

On Windows, dedendent on your compiler, you will either have makefiles for the test cases or a separate Visual Studio project to run the tests.

Have fun! 